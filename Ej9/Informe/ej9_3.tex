\begin{document}
	\chapter{Ejercicio 9}
	\section{Ejercicio 9.3}
	\subsection{Filtros FIR}
	Los filtros FIR (Finite Impulse Response) son filtros de respuesta impulsiva finita. El filtro de la consigna resulta ser un filtro FIR, y en particular nos interesa estudiar aquellos filtros FIR sin distorsión de fase, es decir, aquellos filtros cuya fase sea lineal en función de la frecuencia, o cuyo retardo de grupo sea constante. \par
			\subsection{Filtro de fase lineal}
 	Los filtros FIR de fase lineal se dividen en cuatro tipos básicos según la forma de la respuesta impulsiva que los caracteriza. 
 	\subsection{Tipo I}
 	\subsection{Tipo II}
 	\subsection{Tipo III}
 	\subsection{Tipo IV}

	\subsection{Caso particular}
	
		Procedemos a demostrar por qué un filtro de tipo I es un filtro de fase lineal:\par
		
		Sea $h(n)\epsilon \mathbb{R} / h(n) = h(-n)$, con $n\epsilon \mathbb{Z}$. Entonces, su transformada de fourier $H(f)$ será $H(f)\epsilon \mathbb{R} / H(f) = H(-f)$, por lo que $|H(f)|\epsilon \mathbb{R} / |H(f)| = |H(-f)|$. 
		Sea entonces un filtro discreto realizable de la forma $y(n) = $\sum_{n=0}^{m} a_{j}\cdot x(n-j)$$, con $a_j \epsilon \mathbb{R}$, $m\epsilon \mathbb{N}$ de manera tal que $a_j = a_{m-j}$. La respuesta impulsiva $a(n)$ del filtro estará dada entonces por sus coeficientes, que describen una función simétrica con respejo algún valor $k\epsilon\mathbb{N}$. Luego, $a(n) = h(n-k)$, $h(n)$ par. Entonces, si $A(f)$ es la transformada de fourier de $a(n)$, notamos que $A(f) = H(f)\cdot e^{i2\pi f\cdot k}$ usando la propiedad de traslación en el tiempo para la transformada de fourier. De aquí obtenemos las siguientes propiedades para $A(f)$: \par
		\begin{itemize}
			\item $|A(f)| = |H(f)|$, función real y par.
			\item Como H(f) es una función real, entonces la fase de A(f) será lineal con respecto a la frecuencia y por lo tanto su retardo de grupo $\tau_{g}$ será constante. 
		\end{itemize}  
		
		
		\subsection{Filtro particular}
		Dado el filtro de consigna descripto por la ecuación 
		
			
\end{document}